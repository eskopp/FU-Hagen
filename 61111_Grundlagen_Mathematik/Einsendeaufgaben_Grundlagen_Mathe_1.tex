\documentclass[%
    12pt,
    a4paper,
    ngerman,
    headheight=29.1pt,
]{scrartcl}

\usepackage[a4paper, left=2.5cm, right=2.5cm, top=3.0cm, bottom=2.5cm]{geometry}

\usepackage[utf8]{inputenc}
\usepackage[T1]{fontenc}

\usepackage[ngerman]{babel}
\usepackage[headsepline]{scrlayer-scrpage}
\usepackage[active]{srcltx}
\usepackage{algorithm}
\usepackage[noend]{algorithmic}
\usepackage{amsmath}
\usepackage{amssymb}
\usepackage{amsthm}
\usepackage{bbm}
\usepackage{enumerate}
\usepackage{graphicx}
\usepackage{ifthen}
\usepackage{listings}
\usepackage{struktex}
\usepackage{xcolor}

\usepackage{hyperref} % <-- ALWAYS include me last!

%%%%%%%%%%%%%%%%%%%%%%%%%%%%%%%%%%%%%%%%%%%%%%%%%%%%%%%%%%%%%%%%%%%%%%%%%%%%%%%%
%%%%%%%%%%%%%% EDIT THIS PART %%%%%%%%%%%%%%%%%%%%%%%%%%%%%%%%%%%%%%%%%%%%%%%%%%
%%%%%%%%%%%%%%%%%%%%%%%%%%%%%%%%%%%%%%%%%%%%%%%%%%%%%%%%%%%%%%%%%%%%%%%%%%%%%%%%

\newcommand{\Fach}{Mathematische Grundlagen - 61111}
\newcommand{\Name}{Erik Skopp}
\newcommand{\Seminargruppe}{FU-Hagen}
\newcommand{\Matrikelnummer}{3301729}
\newcommand{\Semester}{SoSe 2024}
\newcommand{\Uebungsblatt}{1} %  <-- UPDATE ME FOR EACH SHEET

%%%%%%%%%%%%%%%%%%%%%%%%%%%%%%%%%%%%%%%%%%%%%%%%%%%%%%%%%%%%%%%%%%%%%%%%%%%%%%%%
%%%%%%%%%%%%%%%%%%%%%%%%%%%%%%%%%%%%%%%%%%%%%%%%%%%%%%%%%%%%%%%%%%%%%%%%%%%%%%%%

\setlength{\parindent}{0em}

%%%%%%%%%%%%%%%
%% Aufgaben-COMMAND
\newcommand{\Aufgabe}[1]{
  {
  \vspace*{0.5cm}
  \textsf{\textbf{Aufgabe #1}}
  \vspace*{0.2cm}
  }
}
\newcommand{\Loesung}[1]{
  {
  \vspace*{0.5cm}
  \textsf{\textbf{Lösung #1}}
  \vspace*{0.2cm}
  }
}
%%%%%%%%%%%%%%
\hypersetup{
    pdftitle={\Fach{}: Uebungsblatt \Uebungsblatt{}},
    pdfauthor={\Name},
    pdfborder={0 0 0}
}

\lstset{ %
language=java,
basicstyle=\footnotesize\ttfamily,
showtabs=false,
tabsize=2,
captionpos=b,
breaklines=true,
extendedchars=true,
showstringspaces=false,
flexiblecolumns=true,
}

\title{FU-Haegen - \Fach{} - \Uebungsblatt{}}
\author{\Name{}}

\begin{document}
\pagestyle{scrheadings}
\lohead{\sffamily \Fach{} \\ \small \Name{} - \Matrikelnummer{}}
\rohead{\sffamily \Semester{} \\   \Seminargruppe{}}
\pagestyle{plain}
\thispagestyle{scrheadings}
\addtokomafont{pagehead}{\normalfont}
\vspace*{0.2cm}
\begin{center}
\LARGE \sffamily \textbf{FU-Hagen - Mathematische Grundlagen - \Uebungsblatt{}}
\end{center}
\vspace*{0.2cm}

%%%%%%%%%%%%%%%%%%%%%%%%%%%%%%%%%%%%%%%%%%%%%%%%%%%%%%
%% Insert your solutions here %%%%%%%%%%%%%%%%%%%%%%%%
%%%%%%%%%%%%%%%%%%%%%%%%%%%%%%%%%%%%%%%%%%%%%%%%%%%%%%

\Aufgabe{1.1}\\
\begin{enumerate}
    \item Seien \( A \), \( B \) und \( C \) Aussagen. Beweisen Sie, dass folgende Aussagen logisch äquivalent sind:
    \begin{enumerate}
        \item \( (A \land B) \Rightarrow C \) und \( (A \Rightarrow C) \lor (B \Rightarrow C) \) sind äquivalent.
        \item \( (A \lor B) \Rightarrow C \) und \( (A \Rightarrow C) \land (B \Rightarrow C) \) sind äquivalent.
    \end{enumerate}
\end{enumerate}

\Loesung{1.1} \\
Um die logische Äquivalenz zu zeigen, betrachten wir die Wahrheitstabelle für beide Ausdrücke.

\begin{enumerate}
    \item[(a)] Um die logische Äquivalenz zu zeigen, betrachten wir die Wahrheitstabelle für beide Ausdrücke.
    
    Die Wahrheitstabelle für die beiden Ausdrücke lautet wie folgt:
    
    \[
    \begin{array}{|c|c|c|c|c|}
    \hline
    A & B & C & (A \land B) \Rightarrow C & (A \Rightarrow C) \lor (B \Rightarrow C) \\
    \hline
    \hline
    0 & 0 & 0 & 1 & 1 \\
    0 & 0 & 1 & 1 & 1 \\
    0 & 1 & 0 & 1 & 1 \\
    0 & 1 & 1 & 1 & 1 \\
    1 & 0 & 0 & 1 & 1 \\
    1 & 0 & 1 & 0 & 1 \\
    1 & 1 & 0 & 0 & 1 \\
    1 & 1 & 1 & 1 & 1 \\
    \hline
    \end{array}
    \]
    
    Aus der Wahrheitstabelle sehen wir, dass die Ausdrücke in beiden Spalten für alle möglichen Kombinationen von Wahrheitswerten von \( A \), \( B \) und \( C \) übereinstimmen. Daher sind die Ausdrücke logisch äquivalent.
    
    \item[(b)] Um die logische Äquivalenz zu zeigen, betrachten wir die Wahrheitstabelle für beide Ausdrücke.
    
    Die Wahrheitstabelle für die beiden Ausdrücke lautet wie folgt:
    
    \[
    \begin{array}{|c|c|c|c|c|}
    \hline
    A & B & C & (A \lor B) \Rightarrow C & (A \Rightarrow C) \land (B \Rightarrow C) \\
    \hline
    \hline
    0 & 0 & 0 & 1 & 1 \\
    0 & 0 & 1 & 1 & 1 \\
    0 & 1 & 0 & 1 & 1 \\
    0 & 1 & 1 & 1 & 1 \\
    1 & 0 & 0 & 1 & 1 \\
    1 & 0 & 1 & 0 & 1 \\
    1 & 1 & 0 & 0 & 1 \\
    1 & 1 & 1 & 1 & 1 \\
    \hline
    \end{array}
    \]
    
    Auch hier sehen wir, dass die Ausdrücke in beiden Spalten für alle möglichen Kombinationen von Wahrheitswerten von \( A \), \( B \) und \( C \) übereinstimmen. Daher sind die Ausdrücke logisch äquivalent.
\end{enumerate}

\Aufgabe{1.2} \\
Beweisen Sie folgende Aussagen mit vollständiger Induktion:

\begin{enumerate}
    \item[(a)] Für alle \( n \in \mathbb{N} \) gilt:
    \[
    \sum_{k=1}^{n} \frac{1}{(2k-1)(2k+1)} = \frac{n}{2n+1}
    \]

    \item[(b)] Für alle \( n \in \mathbb{N} \) mit \( n \geq 2 \) gilt \( n^3 > n^2 + 1 \).
\end{enumerate}

\Loesung{1.2} \\
\begin{enumerate}
    \item[(a)] Vollständige Induktion für eine Summenformel
    
    \textbf{Behauptung:}
    \[
    \sum_{k=1}^{n} \frac{1}{(2k-1)(2k+1)} = \frac{n}{2n+1}
    \]

    \textbf{Beweis:}

    \textit{Induktionsanfang:}
    Setzen wir \(n=1\):
    \[
    \sum_{k=1}^{1} \frac{1}{(2k-1)(2k+1)} = \frac{1}{1 \cdot 3} = \frac{1}{3}
    \]
    Die rechte Seite der Gleichung gibt:
    \[
    \frac{1}{2 \cdot 1 + 1} = \frac{1}{3}
    \]
    Die Gleichung hält für \(n=1\).

    \textit{Induktionsschritt:}
    \[
    \text{Induktionsvoraussetzung:} \quad \sum_{k=1}^{n} \frac{1}{(2k-1)(2k+1)} = \frac{n}{2n+1}
    \]
    \textit{Induktionsbehauptung:}
    \[
    \sum_{k=1}^{n+1} \frac{1}{(2k-1)(2k+1)} = \frac{n+1}{2(n+1)+1}
    \]
    \textit{Induktionsschluss:}
    \[
    \sum_{k=1}^{n+1} \frac{1}{(2k-1)(2k+1)} = \sum_{k=1}^{n} \frac{1}{(2k-1)(2k+1)} + \frac{1}{(2n+1)(2n+3)}
    \]
    \[
    = \frac{n}{2n+1} + \frac{1}{(2n+1)(2n+3)}
    \]
    \[
    = \frac{n(2n+3) + 1}{(2n+1)(2n+3)} = \frac{2n^2 + 3n + 1}{2n^2 + 5n + 3} = \frac{n+1}{2n+3}
    \]
    Die Induktion ist abgeschlossen.

    \item[(b)] Nachweis einer Ungleichung
    
    \textbf{Behauptung:}
    \[
    n^3 > n^2 + 1 \quad \text{für alle } n \geq 2
    \]

    \textbf{Beweis:}

    \textit{Induktionsanfang:}
    Setzen wir \(n=2\):
    \[
    2^3 = 8 > 5 = 2^2 + 1
    \]
    Die Ungleichung hält für \(n=2\).

    \textit{Induktionsschritt:}
    \[
    \text{Induktionsvoraussetzung:} \quad n^3 > n^2 + 1
    \]
    \[
    \textit{Induktionsbehauptung:} \quad (n+1)^3 > (n+1)^2 + 1
    \]
    \textit{Induktionsschluss:}
    \[
    (n+1)^3 = n^3 + 3n^2 + 3n + 1 > n^2 + 1 + 3n^2 + 3n + 1
    \]
    \[
    = 4n^2 + 3n + 2 > (n+1)^2 + 1 = n^2 + 2n + 1 + 1
    \]
    Da \(4n^2 + 3n + 2\) offensichtlich größer als \(n^2 + 2n + 2\) ist, ist die Induktion abgeschlossen.
\end{enumerate}

\Aufgabe{1.3}\\
Bestimmen Sie alle Matrizen \( A \in M_{2 \times 2}(\mathbb{R}) \), für die gilt:
\[
A \begin{pmatrix}
1 & 1 \\
0 & 1
\end{pmatrix}
=
\begin{pmatrix}
1 & 1 \\
0 & 1
\end{pmatrix}
A
\]
\Loesung{1.3} \\
Setzen wir \( A \) als eine allgemeine \( 2 \times 2 \)-Matrix:
\[
A = \begin{pmatrix}
a & b \\
c & d
\end{pmatrix}
\]

Multiplizieren wir \( A \) mit der gegebenen Matrix von rechts:
\[
A \begin{pmatrix}
1 & 1 \\
0 & 1
\end{pmatrix}
=
\begin{pmatrix}
a & b \\
c & d
\end{pmatrix}
\begin{pmatrix}
1 & 1 \\
0 & 1
\end{pmatrix}
=
\begin{pmatrix}
a & a+b \\
c & c+d
\end{pmatrix}
\]

Und von links:
\[
\begin{pmatrix}
1 & 1 \\
0 & 1
\end{pmatrix}
A
=
\begin{pmatrix}
1 & 1 \\
0 & 1
\end{pmatrix}
\begin{pmatrix}
a & b \\
c & d
\end{pmatrix}
=
\begin{pmatrix}
a+c & b+d \\
c & d
\end{pmatrix}
\]

Setzen wir die beiden Resultate gleich:
\[
\begin{pmatrix}
a & a+b \\
c & c+d
\end{pmatrix}
=
\begin{pmatrix}
a+c & b+d \\
c & d
\end{pmatrix}
\]

Daraus ergeben sich die Gleichungen:
\begin{align*}
a &= a+c, \\
a+b &= b+d, \\
c &= c, \\
c+d &= d.
\end{align*}

Diese Gleichungen vereinfachen sich zu:
\begin{align*}
c &= 0, \\
b &= d.
\end{align*}

Die allgemeine Form der Matrix \( A \), die die gegebene Bedingung erfüllt, ist daher:
\[
A = \begin{pmatrix}
a & b \\
0 & b
\end{pmatrix}
\]

\Aufgabe{1.4} \\
Sei \( f : \mathbb{Z} \to \mathbb{Z} \) definiert durch \( f(z) = |z| \) für alle \( z \in \mathbb{Z} \). Dabei ist \( |z| = z \), falls \( z \geq 0 \), und \( |z| = -z \), falls \( z < 0 \).

\begin{enumerate}
    \item[(a)] Untersuchen Sie, ob \( f \) surjektiv beziehungsweise injektiv ist.
    \item[(b)] Sei \( U = \{-3, -2, -1, 0, 1, 2, 3\} \). Bestimmen Sie \( f(U) := \{ f(u) | u \in U \} \), und bestimmen Sie die Menge der Urbilder der Elemente in \( f(U) \).
    \item[(c)] Sei \( V = \{-10, -5, 0, 10, 15\} \). Sei \( W \) die Menge der Urbilder der Elemente in \( V \) unter \( f \). Bestimmen Sie die Elemente in \( W \) und in \( f(W) := \{ f(w) | w \in W \} \).
\end{enumerate}

\Loesung{1.4}\\
\begin{itemize}
    \item[(a)] Untersuchung auf Injektivität und Surjektivität
    
    \textbf{Injektivität:}
    Eine Funktion ist injektiv, wenn unterschiedliche Elemente des Definitionsbereichs auf unterschiedliche Elemente des Wertebereichs abgebildet werden. Für \( f \), da \( f(-1) = f(1) = 1 \), ist \( f \) nicht injektiv.
    
    \textbf{Surjektivität:}
    Eine Funktion ist surjektiv, wenn jeder Wert des Wertebereichs mindestens einmal als Funktionswert vorkommt. Da jeder nichtnegative ganze Wert \( y \) in \( \mathbb{Z} \) als \( f(y) \) oder \( f(-y) \) auftritt, ist \( f \) surjektiv.

    \item[(b)] Bestimmung von \( f(U) \) und Urbildern
    
    \textbf{Gegeben:} \( U = \{-3, -2, -1, 0, 1, 2, 3\} \)
    
    \textbf{Berechnung von \( f(U) \):}
    \[
    f(U) = \{ |u| \,|\, u \in U \} = \{0, 1, 2, 3\}
    \]
    
    \textbf{Bestimmung der Urbilder:}
    \begin{itemize}
        \item \( f^{-1}(\{0\}) = \{0\} \)
        \item \( f^{-1}(\{1\}) = \{-1, 1\} \)
        \item \( f^{-1}(\{2\}) = \{-2, 2\} \)
        \item \( f^{-1}(\{3\}) = \{-3, 3\} \)
    \end{itemize}

    \item[(c)]Bestimmung von \( W \) und \( f(W) \)
    
    \textbf{Gegeben:} \( V = \{-10, -5, 0, 10, 15\} \)
    
    \textbf{Bestimmung von \( W \):}
    \[
    W = f^{-1}(V) = \{-15, -10, -5, 0, 5, 10, 15\}
    \]
    
    \textbf{Berechnung von \( f(W) \):}
    \[
    f(W) = \{ |w| \,|\, w \in W \} = \{0, 5, 10, 15\}
    \]
\end{itemize}


\Aufgabe{1.5} \\
Seien \( f : X \rightarrow Y \) und \( g : Y \rightarrow Z \) Abbildungen.

\begin{enumerate}
    \item[(a)] Zeigen Sie: Ist \( g \circ f \) injektiv, dann ist \( f \) injektiv.
    \item[(b)] Geben Sie ein Beispiel für Mengen \( X, Y, Z \) und Abbildungen \( f : X \rightarrow Y \) und \( g : Y \rightarrow Z \), so dass \( g \circ f \) injektiv und \( g \) nicht injektiv ist.
    \item[(c)] Zeigen Sie: Ist \( g \circ f \) surjektiv, dann ist \( g \) surjektiv.
    \item[(d)] Geben Sie ein Beispiel für Mengen \( X, Y, Z \) und Abbildungen \( f : X \rightarrow Y \) und \( g : Y \rightarrow Z \), so dass \( g \circ f \) surjektiv und \( f \) nicht surjektiv ist.
\end{enumerate}

\Loesung{1.5}\\
\begin{itemize}
    \item[(a)] \textbf{Injektivität von \( g \circ f \)}
    
    \textbf{Behauptung:}
    Ist \( g \circ f \) injektiv, dann ist \( f \) injektiv.

    \textbf{Beweis:}
    Angenommen, \( g \circ f \) ist injektiv. Nehmen wir weiter an, dass \( f(x_1) = f(x_2) \) für \( x_1, x_2 \in X \). Dann gilt:
    \[
    g(f(x_1)) = g(f(x_2))
    \]
    Da \( g \circ f \) injektiv ist, folgt \( x_1 = x_2 \). Also ist \( f \) injektiv.

    \item[(b)] \textbf{Beispiel für injektive \( g \circ f \) und nicht injektive \( g \)}
    
    \textbf{Beispiel:}
    Setzen wir \( X = Y = Z = \mathbb{R} \), \( f(x) = x^2 \) und \( g(y) = \sqrt{y} \) auf \( Y = [0, \infty) \). Hier ist \( g \circ f(x) = |x| \), welches injektiv über \( X = [0, \infty) \) ist, aber \( g(y) = \sqrt{y} \) ist nicht injektiv über \( \mathbb{R} \) wegen \( g(-1) \neq g(1) \) aber \( \sqrt{1} = \sqrt{1} \).

    \item[(c)] \textbf{Surjektivität von \( g \circ f \)}
    
    \textbf{Behauptung:}
    Ist \( g \circ f \) surjektiv, dann ist \( g \) surjektiv.

    \textbf{Beweis:}
    Angenommen, \( g \circ f \) ist surjektiv. Dann gibt es zu jedem \( z \in Z \) ein \( x \in X \), so dass \( g(f(x)) = z \). Setzen wir \( y = f(x) \), existiert für jedes \( z \) ein \( y \in Y \), sodass \( g(y) = z \). Folglich ist \( g \) surjektiv.
\end{itemize}

\Aufgabe{1.6} \\
Sei \( K \) ein Körper, und sei \( A \in M_{n \times n}(K) \) eine Matrix, sodass \( AB = BA \) für alle \( B \in M_{n \times n}(K) \) gilt. Beweisen Sie, dass das genau dann der Fall ist, wenn \( A = aI_n \) für ein \( a \in K \) gilt.

\Loesung{1.6}\\
\begin{itemize}
    \item \textbf{Behauptung:}
    Beweisen Sie, dass das genau dann der Fall ist, wenn \( A = aI_n \) für ein \( a \in K \) gilt.
    
    \item \textbf{Beweis:}
    Wir zeigen, dass wenn \( A \) mit jeder Matrix \( B \) in \( M_{n \times n}(K) \) kommutiert, \( A \) ein skalares Vielfaches der Einheitsmatrix sein muss.

    Da \( A \) mit jeder Matrix \( B \) kommutiert, betrachten wir insbesondere die Kommutierung mit den Matrizen \( E_{ij} \) (die Matrizen mit einer 1 an der Stelle \( (i, j) \) und 0 sonst). Für alle \( i \neq j \), gilt:
    \[
    AE_{ij} = E_{ij}A
    \]
    Dies impliziert, dass \( A \) auf der Diagonalen konstant sein muss und außerhalb der Diagonalen Nullen haben muss. Betrachten wir speziell \( B = E_{ii} \) und \( B = E_{jj} \) für \( i \neq j \), sehen wir, dass alle Diagonalelemente gleich sein müssen:
    \[
    AE_{ii} = E_{ii}A \quad \text{und} \quad AE_{jj} = E_{jj}A
    \]
    Daraus folgt, dass \( A = aI_n \) für ein \( a \in K \), weil dies die einzige Form ist, die mit jeder möglichen \( B \) kommutiert.
\end{itemize}

%%%%%%%%%%%%%%%%%%%%%%%%%%%%%%%%%%%%%%%%%%%%%%%%%%%%%%
%%%%%%%%%%%%%%%%%%%%%%%%%%%%%%%%%%%%%%%%%%%%%%%%%%%%%%
\end{document}
