Um die logische Äquivalenz zu zeigen, betrachten wir die Wahrheitstabelle für beide Ausdrücke.

\begin{enumerate}
    \item[(a)] Um die logische Äquivalenz zu zeigen, betrachten wir die Wahrheitstabelle für beide Ausdrücke.
    
    Die Wahrheitstabelle für die beiden Ausdrücke lautet wie folgt:
    
    \[
    \begin{array}{|c|c|c|c|c|}
    \hline
    A & B & C & (A \land B) \Rightarrow C & (A \Rightarrow C) \lor (B \Rightarrow C) \\
    \hline
    \hline
    0 & 0 & 0 & 1 & 1 \\
    0 & 0 & 1 & 1 & 1 \\
    0 & 1 & 0 & 1 & 1 \\
    0 & 1 & 1 & 1 & 1 \\
    1 & 0 & 0 & 1 & 1 \\
    1 & 0 & 1 & 0 & 1 \\
    1 & 1 & 0 & 0 & 1 \\
    1 & 1 & 1 & 1 & 1 \\
    \hline
    \end{array}
    \]
    
    Aus der Wahrheitstabelle sehen wir, dass die Ausdrücke in beiden Spalten für alle möglichen Kombinationen von Wahrheitswerten von \( A \), \( B \) und \( C \) übereinstimmen. Daher sind die Ausdrücke logisch äquivalent.
    
    \item[(b)] Um die logische Äquivalenz zu zeigen, betrachten wir die Wahrheitstabelle für beide Ausdrücke.
    
    Die Wahrheitstabelle für die beiden Ausdrücke lautet wie folgt:
    
    \[
    \begin{array}{|c|c|c|c|c|}
    \hline
    A & B & C & (A \lor B) \Rightarrow C & (A \Rightarrow C) \land (B \Rightarrow C) \\
    \hline
    \hline
    0 & 0 & 0 & 1 & 1 \\
    0 & 0 & 1 & 1 & 1 \\
    0 & 1 & 0 & 1 & 1 \\
    0 & 1 & 1 & 1 & 1 \\
    1 & 0 & 0 & 1 & 1 \\
    1 & 0 & 1 & 0 & 1 \\
    1 & 1 & 0 & 0 & 1 \\
    1 & 1 & 1 & 1 & 1 \\
    \hline
    \end{array}
    \]
    
    Auch hier sehen wir, dass die Ausdrücke in beiden Spalten für alle möglichen Kombinationen von Wahrheitswerten von \( A \), \( B \) und \( C \) übereinstimmen. Daher sind die Ausdrücke logisch äquivalent.
\end{enumerate}