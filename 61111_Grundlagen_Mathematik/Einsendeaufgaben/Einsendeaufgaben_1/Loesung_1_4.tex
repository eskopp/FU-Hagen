\begin{itemize}
    \item[(a)] Untersuchung auf Injektivität und Surjektivität
    
    \textbf{Injektivität:}
    Eine Funktion ist injektiv, wenn unterschiedliche Elemente des Definitionsbereichs auf unterschiedliche Elemente des Wertebereichs abgebildet werden. Für \( f \), da \( f(-1) = f(1) = 1 \), ist \( f \) nicht injektiv.
    
    \textbf{Surjektivität:}
    Eine Funktion ist surjektiv, wenn jeder Wert des Wertebereichs mindestens einmal als Funktionswert vorkommt. Da jeder nichtnegative ganze Wert \( y \) in \( \mathbb{Z} \) als \( f(y) \) oder \( f(-y) \) auftritt, ist \( f \) surjektiv.

    \item[(b)] Bestimmung von \( f(U) \) und Urbildern
    
    \textbf{Gegeben:} \( U = \{-3, -2, -1, 0, 1, 2, 3\} \)
    
    \textbf{Berechnung von \( f(U) \):}
    \[
    f(U) = \{ |u| \,|\, u \in U \} = \{0, 1, 2, 3\}
    \]
    
    \textbf{Bestimmung der Urbilder:}
    \begin{itemize}
        \item \( f^{-1}(\{0\}) = \{0\} \)
        \item \( f^{-1}(\{1\}) = \{-1, 1\} \)
        \item \( f^{-1}(\{2\}) = \{-2, 2\} \)
        \item \( f^{-1}(\{3\}) = \{-3, 3\} \)
    \end{itemize}

    \item[(c)]Bestimmung von \( W \) und \( f(W) \)
    
    \textbf{Gegeben:} \( V = \{-10, -5, 0, 10, 15\} \)
    
    \textbf{Bestimmung von \( W \):}
    \[
    W = f^{-1}(V) = \{-15, -10, -5, 0, 5, 10, 15\}
    \]
    
    \textbf{Berechnung von \( f(W) \):}
    \[
    f(W) = \{ |w| \,|\, w \in W \} = \{0, 5, 10, 15\}
    \]
\end{itemize}
