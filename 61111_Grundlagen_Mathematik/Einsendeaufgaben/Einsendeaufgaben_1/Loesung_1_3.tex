Setzen wir \( A \) als eine allgemeine \( 2 \times 2 \)-Matrix:
\[
A = \begin{pmatrix}
a & b \\
c & d
\end{pmatrix}
\]

Multiplizieren wir \( A \) mit der gegebenen Matrix von rechts:
\[
A \begin{pmatrix}
1 & 1 \\
0 & 1
\end{pmatrix}
=
\begin{pmatrix}
a & b \\
c & d
\end{pmatrix}
\begin{pmatrix}
1 & 1 \\
0 & 1
\end{pmatrix}
=
\begin{pmatrix}
a & a+b \\
c & c+d
\end{pmatrix}
\]

Und von links:
\[
\begin{pmatrix}
1 & 1 \\
0 & 1
\end{pmatrix}
A
=
\begin{pmatrix}
1 & 1 \\
0 & 1
\end{pmatrix}
\begin{pmatrix}
a & b \\
c & d
\end{pmatrix}
=
\begin{pmatrix}
a+c & b+d \\
c & d
\end{pmatrix}
\]

Setzen wir die beiden Resultate gleich:
\[
\begin{pmatrix}
a & a+b \\
c & c+d
\end{pmatrix}
=
\begin{pmatrix}
a+c & b+d \\
c & d
\end{pmatrix}
\]

Daraus ergeben sich die Gleichungen:
\begin{align*}
a &= a+c, \\
a+b &= b+d, \\
c &= c, \\
c+d &= d.
\end{align*}

Diese Gleichungen vereinfachen sich zu:
\begin{align*}
c &= 0, \\
b &= d.
\end{align*}

Die allgemeine Form der Matrix \( A \), die die gegebene Bedingung erfüllt, ist daher:
\[
A = \begin{pmatrix}
a & b \\
0 & b
\end{pmatrix}
\]