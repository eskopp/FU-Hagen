Sei \( f : \mathbb{Z} \to \mathbb{Z} \) definiert durch \( f(z) = |z| \) für alle \( z \in \mathbb{Z} \). Dabei ist \( |z| = z \), falls \( z \geq 0 \), und \( |z| = -z \), falls \( z < 0 \).

\begin{enumerate}
    \item[(a)] Untersuchen Sie, ob \( f \) surjektiv beziehungsweise injektiv ist.
    \item[(b)] Sei \( U = \{-3, -2, -1, 0, 1, 2, 3\} \). Bestimmen Sie \( f(U) := \{ f(u) | u \in U \} \), und bestimmen Sie die Menge der Urbilder der Elemente in \( f(U) \).
    \item[(c)] Sei \( V = \{-10, -5, 0, 10, 15\} \). Sei \( W \) die Menge der Urbilder der Elemente in \( V \) unter \( f \). Bestimmen Sie die Elemente in \( W \) und in \( f(W) := \{ f(w) | w \in W \} \).
\end{enumerate}